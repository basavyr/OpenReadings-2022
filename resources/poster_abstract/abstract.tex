\documentclass[a4paper,10pt,english]{article}

\usepackage{graphicx}
\usepackage{mathptmx}
\usepackage{babel}
\usepackage{orstylet}
\usepackage{amsmath}                

\renewcommand{\familydefault}{\rmdefault}
\renewcommand{\figurename}{Fig.} 

\makeatother

\begin{document}

\title{INFLUENCE OF THE MOMENTS OF INERTIA ON THE SHAPE AND DYNAMIC OF A TRIAXIAL NUCLEUS}


\author{\uline{Robert Poenaru}$^{1,2}$}

\maketitle

\address{$^{1}$Doctoral School of Physics, University of Bucharest, Bucharest, Romania}

\address{$^{2}$Department of Theoretical Physics, NIPNE-HH, Magurele, Romania}

\rightaddress{robert.poenaru@drd.unibuc.ro}

The energy spectra of rapidly rotating nuclei have specific rotational spectra, parametrized in terms of some empirical constants called moments of inertia (MOI), in correspondence to the energy levels of a rigid rotating top. In this work, a quantitative description of the moments of inertia (e.g, kinematic, dynamic) is made, by using different formulas for their calculation. A comparison between the moments of inertia specific to the rigid case and irrotational one is also shown for a set of deformation parameters specific for the triaxial shapes (i.e., $\beta$, $\gamma$) \cite{key-0}. Moreover, the connection between the wobbling motion, namely the wobbling frequency, the rotational frequency, and the three MOIs is analyzed for a specific excited spectrum of odd-mass nuclei \cite{key-1}. The different modes of rotation for the deformed rigid bodies are also specified in terms of these quantities. As a final goal, the theoretical MOIs of several odd-$A$ nuclei are compared to the experimental values using the $ab$ formula\cite{key-2}.

\begin{thebibliography}{References}
\bibitem{key-0} Masayuki Matsuzaki, Yoshifumi R. Shimizu, and Kenichi Matsuyanagi , AIP Conference Proceedings 865, 139-144 (2006)

\bibitem{key-1} Masayuki Matsuzaki, Yoshifumi R. Shimizu, and Kenichi Matsuyanagi Phys. Rev. C \textbf{65}, 041303(R)
    
\bibitem{key-2} Timár et al. Phys. Rev. Lett. \textbf{122}, 062501

\end{thebibliography}

\end{document}
